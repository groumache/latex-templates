% Based on:  https://github.com/noraj/OSCP-Exam-Report-Template-Markdown

\documentclass[french,oneside]{article}

%% Language and font encodings
\usepackage[french]{babel}
\usepackage[utf8x]{inputenc}
\usepackage[T1]{fontenc}

%% Useful packages
\usepackage{graphicx}
\usepackage[colorinlistoftodos]{todonotes}
\PassOptionsToPackage{hyphens}{url}
\usepackage[colorlinks=true, allcolors=black]{hyperref}
\usepackage{longtable}

%% Pour changer la police
\usepackage[default]{sourcesanspro}

%% Pour la page de garde
\usepackage{pagecolor}
\usepackage{afterpage}
\usepackage[margin=2.5cm,includehead=true,includefoot=true,centering]{geometry}

%% Pour les exemples
\usepackage{mdframed}
\newmdenv[topline=false, bottomline=false, rightline=false, skipabove=\topsep, skipbelow=\topsep]{example}

%% Tikz
\usepackage{pgf-pie}
\usepackage{tikz}
\usetikzlibrary{backgrounds}
\pgfkeys{/donut/.cd,
inner radius/.initial=3.14cm,
inner radius=3.14cm,
outer radius/.initial=2cm,
outer radius=2cm,
text color/.initial=white,
text color=white}
\newcommand{\donutchart}[2][]{
   % Calculate total
   \pgfmathsetmacro{\totalnum}{0}
   \foreach [count=\n] \value/\colour/\name in {#2} {
     \pgfmathparse{\value+\totalnum}
     \global\let\totalnum=\pgfmathresult
     \xdef\numitems{\n}
   }

  \begin{tikzpicture}
  \pgfmathsetmacro{\wheelwidth}{\pgfkeysvalueof{/donut/outer
  radius}-\pgfkeysvalueof{/donut/inner radius}}
  \pgfmathsetmacro{\midradius}{(\pgfkeysvalueof{/donut/outer radius}
  +\pgfkeysvalueof{/donut/inner radius})/2}

  \begin{scope}[#1]

    \pgfmathsetmacro{\cumnum}{0}
    \foreach \value/\colour/\name in {#2} {
        \pgfmathsetmacro{\newcumnum}{\cumnum + \value/\totalnum*360}

        \pgfmathsetmacro{\midangle}{-(\cumnum+\newcumnum)/2}
        \begin{scope}[on background layer]
          \filldraw[draw=white,fill=\colour]
          (-\cumnum:\pgfkeysvalueof{/donut/outer radius}) 
           arc(-\cumnum:-(\newcumnum):\pgfkeysvalueof{/donut/outer radius}) --
          (-\newcumnum:\pgfkeysvalueof{/donut/inner radius}) 
          arc(-\newcumnum:-(\cumnum):\pgfkeysvalueof{/donut/inner radius}) -- cycle;
        \end{scope}
        \draw node [text=\pgfkeysvalueof{/donut/text color}, 
        font=\bfseries\sffamily] at 
        (\midangle:{\pgfkeysvalueof{/donut/inner radius}+\wheelwidth/2}) {\value \%};

        \global\let\cumnum=\newcumnum
    }
    
    \node[text width=\textwidth] at (-4,0) {
        \foreach \value/\colour/\name in {#2} {
           {\color{\colour}\textbullet}\space\name \hfill
        }    
    };
    
    \node at (0,0) {Vulnérabilités};
    
  \end{scope}

  \end{tikzpicture}}

% variables for title, author and date
\usepackage{titling}
\title{Pentest des infrastructures de MegaCorpOne}
\author{Grégoire Roumache - Maxime Wallemme}
\date{21-12-2021}

%% header & footer
\usepackage[headsepline,footsepline]{scrlayer-scrpage}
\newpairofpagestyles{header-footer}{
  \clearpairofpagestyles
  \ihead[21-12-2021]{\raisebox{1ex}{Pentest des infrastructures de MegaCorpOne}}
  \chead[]{}
  \ohead[Pentest des infrastructures de MegaCorpOne]{\raisebox{1ex}{21-12-2021}}
  \ifoot[\thepage]{\raisebox{-1ex}{Grégoire Roumache - Maxime Wallemme}}
  \cfoot[]{}
  \ofoot[Grégoire Roumache - Maxime Wallemme]{\raisebox{-1ex}{\thepage}}
  \addtokomafont{pageheadfoot}{\upshape}
}
\pagestyle{header-footer}


%% To list and caption code
% \usepackage{minted}
% \renewcommand{\listoflistingscaption}{Table des programmes}
% \usepackage{caption}
% \newenvironment{code}{\captionsetup{type=listing}}{}
% \renewcommand{\listingscaption}{Programme}



\begin{document}




\begin{titlepage}
    \newgeometry{left=6cm}
    \definecolor{titlepage-color}{HTML}{1E90FF}
    \newpagecolor{titlepage-color}\afterpage{\restorepagecolor}
    \newcommand{\colorRule}[3][black]{\textcolor[HTML]{#1}{\rule{#2}{#3}}}
    \begin{flushleft}
        \noindent
        \\[-1em]
        \color[HTML]{FFFAFA}
        \makebox[0pt][l]{\colorRule[FFFAFA]{1.3\textwidth}{2pt}}
        \par
        \noindent
        {
          \vfill
          \noindent {\huge \textbf{\textsf{Pentest des infrastructures de MegaCorpOne}}}
            \vskip 1em
          {\Large \textsf{Rapport de pentest}}
            \vskip 2em
          \noindent {\Large \textsf{Grégoire Roumache - Maxime Wallemme}}
          \vfill
        }
        \textsf{21 Décembre 2021}
    \end{flushleft}
\end{titlepage}

\let\cleardoublepage\clearpage

\tableofcontents \newpage



\begin{center}
\section*{Historique des versions}
\addcontentsline{toc}{section}{Historique des versions}
\end{center}

\begin{table}[h]
    \centering
    \begin{tabular}{|l|c|l|}
        \hline
        Date du document & Version & Description des changements \\
        \hline
        21-12-2021 & 1.0 & Version initiale \\
        \hline
    \end{tabular}
\end{table}


\cleardoublepage
\newpage








\section{Résumé}

Nous avons reçu la tâche de réaliser un pentest des infrastructures de MegaCorpOne. Un pentest est une méthode d'évaluation de la sécurité d'un système d'information par la simulation d'attaques similaires à celles réalisées par des hackers dont l'objectif serait d'infiltrer MegaCorpOne. Les objectifs de ce test étaient d'identifier les infrastructures informatiques de l'entreprise et les failles qu'elles possèdent, de les exploiter, ainsi que de proposer des solutions pour les combler.

Lors de la réalisation de ce pentest, nous avons trouvés plusieurs vulnérabilités critiques dans le système d'information de MegaCorpOne. Nous avons pu accéder à plusieurs machines du réseau de l'entreprise et ce, avec un accès administrateur. Ceci est dû principalement à des configurations de sécurité faibles, des applications qui ne sont pas à jour et une politique de mots de passe trop faible.

Voici une liste des systèmes et le niveau d'accès que nous avons pu obtenir :

\begin{center} \begin{tabular}{llll}
    Adresse IP    & Nom d'hôte & Système d'exploitation & Niveau d'accès            \\ \hline
    10.180.20.1   & ...        & Windows 2016           & Administrateur du domaine \\
    10.180.20.2   & ...        & Windows 2003           & Administrateur            \\
    10.180.20.3   & ...        & Windows 10             & Administrateur            \\
    10.180.20.11  & /          & Linux                  & /                         \\
    10.180.30.10  & ...        & Linux                  & Accès root                \\
    10.180.30.15  & ...        & Linux                  & /                         \\
\end{tabular} \end{center}
Les adresses IP 10.180.20.254 et 10.180.30.254 étaient aussi utilisées mais ce sont les adresses des routeurs des sous-réseaux que nous avons attaqués.

Et voici la liste des service que nous avons découverts sur ces machines :

\begin{center} \begin{longtable}{llllp{8.5cm}}
    Adresse IP   & Port & Protocol & Nom & Informations complémentaires  \\ \hline
    ...          & ...  & ...      & ... & ...
\end{longtable} \end{center}















\section{Recommendations}

Nous recommandons de patcher les vulnérabilités identifiés lors de ce test et de renforcer la politique de mots de passe. Une fois ces tâches effectuées, nous pensons qu'il faut s'assurer qu'un attaquant ne pourra plus exécuter ces attaques en effectuant des tests. Il faudra aussi faire attention aux nouvelles vulnérabilités qui ne manqueront pas d'arriver dans le futur et continuer de protéger les systèmes au quotidien avec une stratégie de patching régulière.















\section{Méthodologie utilisée}

Nous avons utilisé une méthode souvent pratiquée par les hackers expérimentés pour s'attaquer à une entreprise. Elle est divisée en plusieurs étapes en commençant par la recherche d'informations sur l'entreprise afin de découvrir des vulnérabilités. Une fois trouvées, il faut les exploiter pour rentrer dans le système informatique de l'entreprise. Et finalement, on se déplace dans le système en attaquant d'autres machines de l'infrastructure et en essayant de prendre le contrôle de comptes administrateur.















\subsection{Cadre du pentest}

Dans le cadre de ce pentest, il a été décidé préalablement de chercher des informations pouvant conduire à une attaque de phishing ciblée mais de ne pas conduire cette attaque pour se concentrer sur l'aspect technique du système d'information. Concrètement, cela veut dire que nous avons réuni des informations sur les réseaux sociaux pour pouvoir nous faire passer pour un employé, comme le font certains hackers, afin de faire du social engineering mais nous n'avons pas conduit cette attaque.










\subsection{Reconnaissance}

Lors de cette étape très importante du pentest, nous avons cherché un maximum d'informations sur l'entreprise qui peuvent être utiles à des hackers dont l'objectif est de nuire à l'entreprise. L'objectif est de trouver et comprendre la surface d'attaque de l'entreprise autant d'un point de vue technique comme trouver le nombre de serveurs, domaines et sous-domaines internet; que des données qui pourraient être utiles à des fins de social engineering, comme des informations sur le CEO ou d'autres employés de l'entreprise. Ceci, sans outil de scan et énumération.



\subsubsection{Informations sur quelques employés de la société}

\begin{enumerate}
    \item CEO, Joe Sheer:
    \begin{itemize}
        \item né en 1968,
        \item twitter: \url{https://twitter.com/joe\_sheer/},
        \item email: \url{joe@megacorpone.com}.
    \end{itemize}
    \item Personne de contact technique (IT and Security Director), Alan Grofield:
    \begin{itemize}
        \item travaille chez MegaCorp One depuis 1983,
        \item linkedin: \url{https://www.linkedin.com/in/alan-grofield-32806468},
        \item email: \url{agrofield@megacorpone.com}.
    \end{itemize}
    \item Recruteur, Handy McKay:
    \begin{itemize}
        \item twitter: \url{https://twitter.com/McKayHandy}.
    \end{itemize}
    \item Stagiaire, William Adler:
    \begin{itemize}
        \item twitter: \url{https://twitter.com/RealWillAdler}.
    \end{itemize}
    \item WEB designer, Tom Hudson:
    \begin{itemize}
        \item email: \url{thudson@megacorpone.com},
        \item twitter: \url{https://twitter.com/TomHudsonMCO}.
    \end{itemize}
    \item Développeur senior (Senior Developper), Tanya Rivera:
    \begin{itemize}
        \item email: \url{trivera@megacorpone.com},
        \item twitter: \url{https://twitter.com/TanyaRiveraMCO}.
    \end{itemize}
    \item Directeur marketing (Marketing Director), Matt Smith:
    \begin{itemize}
        \item email: \url{msmith@megacorpone.com},
        \item twitter: \url{https://twitter.com/MattSmithMCO}.
    \end{itemize}
    \item Vice-président des affaires juridiques (VP Of Legal), Mike Carlow:
    \begin{itemize}
        \item email: \url{mcarlow@megacorpone.com},
        \item linkedin: \url{https://www.linkedin.com/in/mike-carlow-8128896a/}.
    \end{itemize}
\end{enumerate}



\subsubsection{Informations sur le nom de domaine}

\begin{enumerate}
    \item Date d'enregistrement du nom de domaine:
    \begin{itemize}
        \item 22-01-2013
    \end{itemize}
    \item Sous-domaines de megacorpone:
    \begin{itemize}
        \item admin.megacorpone.com
        \item beta.megacorpone.com
        \item fs1.megacorpone.com
        \item ftp.megacorpone.com
        \item intranet.megacorpone.com
        \item mail.megacorpone.com
        \item mail2.megacorpone.com
        \item manager.megacorpone.com
        \item mgmt.megacorpone.com
        \item michael.megacorpone.com
        \item ns1.megacorpone.com
        \item ns2.megacorpone.com
        \item ns3.megacorpone.com
        \item remote.megacorpone.com
        \item router.megacorpone.com
        \item siem.megacorpone.com
        \item snmp.megacorpone.com
        \item support.megacorpone.com
        \item svn.megacorpone.com
        \item syslog.megacorpone.com
        \item test.megacorpone.com
        \item vpn.megacorpone.com
        \item webmail.megacorpone.com
        \item www.megacorpone.com
        \item www2.megacorpone.com
    \end{itemize}
\end{enumerate}



\subsubsection{Informations sur le serveur WEB et les serveurs DNS}

\begin{enumerate}
    \item Hébergeur:
    \begin{itemize}
        \item OVH
    \end{itemize}
    \item Pays d'hébergement du site web:
    \begin{itemize}
        \item Montréal, Québec, Canada
    \end{itemize}
    \item Adresse IP publique du site web:
    \begin{itemize}
        \item 149.56.244.87
    \end{itemize}
    \item Changement d'hébergement du site web:
    \begin{itemize}
        \item Le site web n'a pas changé d'hébergement. Le statut du transfert de client est sur "interdit" (clientTransferProhibited).
    \end{itemize}
    \item Operating system du site web: Debian 10
    \item Technologie du web server: apache/2.4.38
    \item Vulnérabilités connues pour la version du web server:
    \begin{itemize}
        \item CVE-2018-17189: DoS for HTTP/2 connections via slow request bodies (low)
        \item CVE-2018-17199: mod\_session\_cookie does not respect expiry time (low)
        \item CVE-2019-0190: mod\_ssl 2.4.37 remote DoS when used with OpenSSL 1.1.1 (important)
    \end{itemize}
    \item Technologies web utilisées: PHP/7.3.29-1
    \item Name servers de la société:
    \begin{itemize}
        \item ns1.megacorpone.com
        \item ns2.megacorpone.com
        \item ns3.megacorpone.com
    \end{itemize}
\end{enumerate}



\subsubsection{Informations sensibles}

\begin{enumerate}
    \item Hash du mot de passe de l'utilisateur trivera, trouvé sur le github de l'entreprise:
    \begin{itemize}
        \item hash: \texttt{trivera{}:\$apr1\$A0vSKwao\$GV3sgGAj53j.c3GkS4oUC0}
        \item commande utilisée pour trouver le mot de passe à partir du hash: \texttt{john -{}-wordlist=rockyou.txt hash.txt}
        \item résultat: \texttt{trivera{}:Tanya4life}
    \end{itemize}
    
    \item Informations sur l'utilisateur William Adler:
    Nous supposons qu'il pourrait s'agir des identifiants de l'utilisateur pouvant lui servir à accéder à l'un des services interne de l'entreprise.
    \begin{itemize}
        \item Nom d'utilisateur: Wadler
        \item Mot de passe: TwitterStar2
        \item Informations trouvées sur le twitter de l'employé William Adler
    \end{itemize}
\end{enumerate}



\subsubsection{Autres informations}

\begin{enumerate}
    \item Adresse de l'entreprise: 2 Old Mill St, 89001 Rachel, Nevada US.
    \item Numéro de téléphone: +1.9038836342.
    \item Adresses mail des différents départements:
    \begin{itemize}
        \item département des ressources humaines (Human Resources): \url{hr@megacorpone.com},
        \item département commercial (Sales): \url{sales@megacorpone.com},
        \item département logistique (Shipping): \url{shipping@megacorpone.com}.
    \end{itemize}
    \item Réseaux sociaux de l'entreprise:
    \begin{itemize}
        \item facebook (contenu indisponible): \url{https://www.facebook.com/MegaCorp-One-393570024393695/}
        \item linkedin (contenu indisponible): \url{https://www.linkedin.com/company/18268898/}
        \item github: \url{https://github.com/megacorpone}
    \end{itemize}
    \item Format d'adresses mails de la société:
    \begin{itemize}
        \item Employés: 1ère lettre du prénom + nom de famille + @megacorpone.com
        \item Département: Initiales ou nom complet du département + @megacorpone.com
        \item PDG: Prénom + @megacorpone.com
    \end{itemize}
    \item Page web non référencée: \url{https://www.megacorpone.com/nanites.php}
    \item Page web anciennement référencées:
    \begin{itemize}
        \item \url{http://www.megacorpone.com/jobs2.html}
        \item \url{https://cp.megacorpone.net/}
    \end{itemize}
\end{enumerate}










\subsection{Scanning et énumération}

Le scanning et l'énumération est la deuxième étape du pentest, elle s'inscrit dans la continuité de l'étape de reconnaissance parce qu'elle continue la recherche d'informations sur l'entreprise mais avec une différence majeure: l'utilisation d'outils de scanning automatisés. L'objectif est de trouver un maximum d'informations sur les services informatiques de l'entreprise accessibles depuis l'extérieur. Par exemple, nous avons scanné le site web de MegaCorpOne pour trouver des pages intéressantes comme une partie du site réservée aux administrateurs.



\subsubsection{Identification des adresses IP d'intérêt}



\subsubsection{Scan du réseau}



\subsubsection{Nmap - TCP scanning}



\subsubsection{Nmap - UDP scanning}



\subsubsection{OS Fingerprinting}

résultats:
\begin{itemize}
    \item 10.180.30.10:  Linux 3.2 - 4.9
    \item 10.180.30.15:  Linux 4.15 - 5.6
    \item 10.180.30.254: Linux 4.15 - 5.6
    \item 10.180.20.1:   Microsoft Windows Server 2016
    \item 10.180.20.2:   Microsoft Windows Server 2003 SP1 or SP2, Microsoft Windows Server 2003 SP2, Microsoft Windows Server 2008 Enterprise SP2
    \item 10.180.20.3:   Microsoft Windows 10 1709 - 1909
    \item 10.180.20.11:  Linux 4.15 - 5.6
    \item 10.180.20.254: Linux 4.15 - 5.6
\end{itemize}



\subsubsection{Énumeration SNMP}



\subsubsection{Énumeration WEB}



\subsubsection{Énumeration Wordpress}












\subsection{Analyse de vulnérabilités}



\subsubsection{Types de vulnérabilités}

Voici les différents types de vulnérabilités trouvés, on remarque que les scores les plus élevés comme Critical, High et Medium sont présents en grand nombre.

\donutchart[rotate=90]{16.7/purple/CRITICAL,23.3/red/HIGH,50/orange/MEDIUM,10/yellow/LOW}



\subsubsection{Hôte - 10.180.20.1}

...



\subsubsection{Hôte - 10.180.20.2}

...










\subsection{Exploitation}



\subsubsection{Accès au réseau Wi-fi}

...



\subsubsection{Service SSH}

...



\subsubsection{...}










\subsection{Élévation de privilèges}

...










\subsection{Mouvement latéral}

...










\subsection{Recommandations de sécurité} \label{mesures}

\begin{enumerate}

\item ...

\end{enumerate}










\cleardoublepage
\newpage


\section{Conclusion}

...

























\appendix \newpage

























\section{Annexes}

...















\newpage \listoffigures % \newpage \listoflistings
\newpage
\begin{thebibliography}{9}
% \bibitem{1} Consulté le 20-11-2021 \url{}
% \bibitem{2} Consulté le 20-11-2021 \url{}
% \bibitem{3} Consulté le 20-11-2021 \url{}
% \bibitem{4} 
% \bibitem{5} 
% \bibitem{6} 
\end{thebibliography}




















\end{document}
