\documentclass[a4paper]{article}

%% Language and font encodings
\usepackage[french]{babel}
\usepackage[utf8x]{inputenc}
\usepackage[T1]{fontenc}

%% Sets page size and margins
\usepackage[a4paper,top=3cm,bottom=3cm,left=2cm,right=2cm,marginparwidth=2cm]{geometry}

%% Useful packages
\usepackage{amsmath}
\usepackage{graphicx}
\usepackage{float}
\usepackage[colorinlistoftodos]{todonotes}
\PassOptionsToPackage{hyphens}{url}
\usepackage[colorlinks=true, allcolors=black]{hyperref}
\usepackage{fourier-orns}
\usepackage{titlesec}
\usepackage{fancyhdr}
\usepackage{fancyvrb}
\pagestyle{fancy}
\setcounter{tocdepth}{5}

%% For examples
\usepackage{mdframed}
\newmdenv[topline=false, bottomline=false, rightline=false, skipabove=\topsep, skipbelow=\topsep]{example}

%% Tikz stuff
\usepackage{tikz}
\usetikzlibrary{calc, arrows}
\tikzstyle{incolore} = [rectangle, rounded corners, draw=black, minimum height=1cm, minimum width=3cm, text width=3cm, text centered]

%% To list and caption code
\usepackage{minted}
\renewcommand{\listoflistingscaption}{Table des programmes}
\usepackage{caption}
\newenvironment{code}{\captionsetup{type=listing}}{}
\renewcommand{\listingscaption}{Programme}

\usepackage{libertine}
\newcommand{\hsp}{\hspace{20pt}}
\newcommand{\HRule}{\rule{\linewidth}{0.5mm}}

\renewcommand{\headrulewidth}{1pt}
\fancyhead[C]{}
\fancyhead[L]{}
\fancyhead[R]{\footnotesize{\leftmark}}

\renewcommand{\footrulewidth}{1pt}
\fancyfoot[C]{}
\fancyhead[L]{}
\fancyfoot[R]{\thepage}

\usepackage{eso-pic,graphicx}
\usepackage{xcolor}
\newcommand{\bgimg}[1] {
    \AddToShipoutPicture {
        \put(\LenToUnit{0 cm},\LenToUnit{0 cm}) {
            \includegraphics[width=\paperwidth,height=\paperheight]{#1}
        }
    }
}
\begin{document}





\begin{titlepage}
    \begin{sffamily}
        \begin{center}

            \includegraphics[width=5cm]{images/LogoHenallux.PNG}~\\[1.5cm]
            \textsc{\Large Rapport de laboratoire}\\[1.5cm]

            \HRule \\[0.4cm]
            { \huge \bfseries Stabilité d’un pont suspendu soumis à des sollicitations cycliques \\[0.4cm] }
            \HRule \\[2cm]

            \begin{minipage}{0.4\textwidth}
                \begin{flushleft} \large
                    Grégoire Roumache \\
                    XXXX XXXX \\
                    XXXX XXXX \\
                    XXXX XXXX \\
                \end{flushleft}
            \end{minipage}
            \begin{minipage}{0.55\textwidth}
                \begin{flushright} \large
                    Laboratoire de sciences appliquées à l'informatique \\
                    Hénallux, Sécurité des systèmes \\
                    Première année, groupe H \\
                    Année académique 2019-2020 \\
                \end{flushright}
            \end{minipage}
            \vfill

            {\large 3 Septembre 2017}

        \end{center}
    \end{sffamily}
\end{titlepage}

\let\cleardoublepage\clearpage










\section{Introduction}



%%










\section{Contenu}



Voici comment référencer une figure \ref{fig:figurename}. Comment référencer un élément de la bibliographie \cite{1}. Et voici pour un programme \ref{code:progpython}.

\begin{figure}[H]
    \centering
    \includegraphics[width=0.3\linewidth]{images/LogoHenallux.PNG}
    \caption{Légende de la figure}
    \label{fig:figurename}
\end{figure}

\begin{code}\small
\begin{minted}[xleftmargin=20pt,linenos]{python}
for i in range(2, 1000):
    if is_prime(i): print(i)
\end{minted}
\caption{Affichage des nombres premiers entre 0 et 1000}
\label{code:progpython}
\end{code}










\section{Conclusion}



%%










\appendix










\section{Annexe}



%%










\newpage \tableofcontents \listoffigures \listoflistings
\begin{thebibliography}{9}
\bibitem{1} {\small \url{http://www.example.com}}
% \bibitem{2} 
% \bibitem{3} 
% \bibitem{4} 
% \bibitem{5} 
% \bibitem{6} 
\end{thebibliography}










\end{document}
